\documentclass[12pt,halfline,a4paper,]{ouparticle}

% Packages I think are necessary for basic Rmarkdown functionality
\usepackage{hyperref}
\usepackage{graphicx}
\usepackage{listings}
\usepackage{color}
\usepackage{fancyvrb}
\usepackage{framed}

% For knitr::kable functionality
\usepackage{booktabs}
\usepackage{longtable}

%% To allow better options for figure placement
%\usepackage{float}

% Packages that are supposedly required by OUP sty file
\usepackage{amssymb, amsmath, geometry, amsfonts, verbatim, endnotes, setspace}

% For code highlighting I think
\DefineVerbatimEnvironment{Highlighting}{Verbatim}{commandchars=\\\{\}}
\definecolor{shadecolor}{RGB}{248,248,248}
\newenvironment{Shaded}{\begin{snugshade}}{\end{snugshade}}
\newcommand{\AlertTok}[1]{\textcolor[rgb]{0.94,0.16,0.16}{#1}}
\newcommand{\AnnotationTok}[1]{\textcolor[rgb]{0.56,0.35,0.01}{\textbf{\textit{#1}}}}
\newcommand{\AttributeTok}[1]{\textcolor[rgb]{0.77,0.63,0.00}{#1}}
\newcommand{\BaseNTok}[1]{\textcolor[rgb]{0.00,0.00,0.81}{#1}}
\newcommand{\BuiltInTok}[1]{#1}
\newcommand{\CharTok}[1]{\textcolor[rgb]{0.31,0.60,0.02}{#1}}
\newcommand{\CommentTok}[1]{\textcolor[rgb]{0.56,0.35,0.01}{\textit{#1}}}
\newcommand{\CommentVarTok}[1]{\textcolor[rgb]{0.56,0.35,0.01}{\textbf{\textit{#1}}}}
\newcommand{\ConstantTok}[1]{\textcolor[rgb]{0.00,0.00,0.00}{#1}}
\newcommand{\ControlFlowTok}[1]{\textcolor[rgb]{0.13,0.29,0.53}{\textbf{#1}}}
\newcommand{\DataTypeTok}[1]{\textcolor[rgb]{0.13,0.29,0.53}{#1}}
\newcommand{\DecValTok}[1]{\textcolor[rgb]{0.00,0.00,0.81}{#1}}
\newcommand{\DocumentationTok}[1]{\textcolor[rgb]{0.56,0.35,0.01}{\textbf{\textit{#1}}}}
\newcommand{\ErrorTok}[1]{\textcolor[rgb]{0.64,0.00,0.00}{\textbf{#1}}}
\newcommand{\ExtensionTok}[1]{#1}
\newcommand{\FloatTok}[1]{\textcolor[rgb]{0.00,0.00,0.81}{#1}}
\newcommand{\FunctionTok}[1]{\textcolor[rgb]{0.00,0.00,0.00}{#1}}
\newcommand{\ImportTok}[1]{#1}
\newcommand{\InformationTok}[1]{\textcolor[rgb]{0.56,0.35,0.01}{\textbf{\textit{#1}}}}
\newcommand{\KeywordTok}[1]{\textcolor[rgb]{0.13,0.29,0.53}{\textbf{#1}}}
\newcommand{\NormalTok}[1]{#1}
\newcommand{\OperatorTok}[1]{\textcolor[rgb]{0.81,0.36,0.00}{\textbf{#1}}}
\newcommand{\OtherTok}[1]{\textcolor[rgb]{0.56,0.35,0.01}{#1}}
\newcommand{\PreprocessorTok}[1]{\textcolor[rgb]{0.56,0.35,0.01}{\textit{#1}}}
\newcommand{\RegionMarkerTok}[1]{#1}
\newcommand{\SpecialCharTok}[1]{\textcolor[rgb]{0.00,0.00,0.00}{#1}}
\newcommand{\SpecialStringTok}[1]{\textcolor[rgb]{0.31,0.60,0.02}{#1}}
\newcommand{\StringTok}[1]{\textcolor[rgb]{0.31,0.60,0.02}{#1}}
\newcommand{\VariableTok}[1]{\textcolor[rgb]{0.00,0.00,0.00}{#1}}
\newcommand{\VerbatimStringTok}[1]{\textcolor[rgb]{0.31,0.60,0.02}{#1}}
\newcommand{\WarningTok}[1]{\textcolor[rgb]{0.56,0.35,0.01}{\textbf{\textit{#1}}}}

% For making Rmarkdown lists
\providecommand{\tightlist}{%
  \setlength{\itemsep}{0pt}\setlength{\parskip}{0pt}}

% Part for setting citation format package: natbib

% Part for setting citation format package: biblatex

% Part for indenting CSL refs
% Pandoc citation processing
% Pandoc header

\begin{document}

\title{Template for Oxford University Press papers}

\author{%
\name{Alice Anonymous}\address{Some Institute of Technology}\email{\href{mailto:alice@example.com}{alice@example.com}}
\and
\name{Bob Security}\address{Another University}\email{\href{mailto:bob@example.com}{bob@example.com}}\thanks{Corresponding author; Email: \href{mailto:bob@example.com}{bob@example.com}}
\and
\name{Cat Memes}\address{Another University}\email{\href{mailto:cat@example.com}{cat@example.com}}
\and
\name{Derek Zoolander}\address{Some Institute of Technology}\email{\href{mailto:derek@example.com}{derek@example.com}}
}

\abstract{This is the abstract.

It consists of two paragraphs.}

\date{2021-05-22}

\keywords{key; dictionary; word}

\maketitle



\hypertarget{abstract}{%
\section{Abstract}\label{abstract}}

Here is an example test chapter.

\hypertarget{introduction}{%
\section{Introduction}\label{introduction}}

Lorem ipsum dolor sit amet, consectetur adipiscing elit. Integer tristique felis nec
interdum gravida. Pellentesque ac ornare nunc. Maecenas nec rutrum nulla. Cras dapibus
suscipit arcu, in condimentum risus semper ut. Quisque congue sem a arcu interdum, quis
blandit purus interdum. Aenean placerat, nibh sed pretium iaculis, massa orci convallis
dolor, non ullamcorper nisl lacus eget mi. Integer at dolor posuere, tincidunt nisi et,
pellentesque odio. Donec eget nisi et felis fringilla suscipit a ac quam. Donec in
dignissim justo, nec pretium libero.

Vivamus tristique, libero sed accumsan faucibus, ex urna porta lacus, at ullamcorper leo
lectus eget velit. Nunc sit amet turpis libero. Nunc egestas, elit et commodo commodo,
ipsum neque consectetur ante, rutrum malesuada ante felis vitae lectus. Pellentesque arcu
ipsum, rhoncus id lobortis ornare, tincidunt eu est. Morbi ullamcorper, lorem id dictum
condimentum, erat nulla pharetra nisi, quis fringilla sapien enim sit amet libero. Ut
pulvinar tempor leo euismod elementum. Suspendisse molestie quis enim ut viverra. Nullam
convallis tempus lorem, sed egestas sem ultrices sed. Fusce sed blandit nisi. Nam
facilisis aliquet porta. Vestibulum commodo lectus molestie magna convallis, ac volutpat
turpis auctor. Praesent mollis metus nec tortor ornare, nec tempor enim elementum.
Vestibulum eget luctus risus, ac molestie felis. Vivamus velit arcu, malesuada quis tellus
id, maximus posuere enim.

\hypertarget{methods}{%
\section{Methods}\label{methods}}

Donec ut risus nec mi sollicitudin varius eget et nisi. Fusce laoreet bibendum ultricies.
Proin tempor laoreet tellus, vitae blandit urna tempor et. Integer finibus pellentesque
leo, eget dignissim mi luctus nec. In posuere vehicula massa sit amet euismod.
Pellentesque ultricies turpis a leo porta bibendum. Integer tempor nibh sed semper
suscipit.

\hypertarget{results}{%
\section{Results}\label{results}}

Quisque a imperdiet leo, vel consequat eros. Integer molestie laoreet eleifend. Donec
tempus non sem eget tristique. Nunc porttitor tortor eu dictum eleifend. Fusce augue ex,
euismod ut leo et, eleifend consectetur risus. Vivamus egestas orci sit amet varius
egestas. Nulla aliquet gravida neque vitae tincidunt. Maecenas ac consequat orci, sed
blandit dui. Duis laoreet nisi a mi venenatis ultricies. Sed pulvinar rutrum imperdiet.
Cras viverra metus in diam mattis dignissim. Fusce vehicula viverra eros ut luctus.

\hypertarget{conclusion}{%
\section{Conclusion}\label{conclusion}}

Nam dictum erat vel lorem dapibus, ac malesuada magna efficitur. Nullam sed diam ac ipsum
facilisis dapibus eu ultricies nisl. Suspendisse potenti. Maecenas suscipit ipsum tortor,
et euismod lectus egestas in. Integer suscipit nulla id enim lobortis faucibus. Quisque
metus justo, rutrum vitae aliquet vulputate, fermentum non sem. Pellentesque posuere velit
quis mauris lacinia, at feugiat neque auctor. Mauris congue mollis accumsan. Ut pharetra
ut arcu interdum gravida. Sed convallis ante at ex rutrum mollis. Nunc pharetra quis felis
sed efficitur.

\hypertarget{introduction-1}{%
\section{Introduction}\label{introduction-1}}

This template is based on the generic OUP template available \href{https://academic.oup.com/icesjms/pages/General_Instructions}{here}. The original OUP sample tex document, providing more details on prefered formatting for LaTeX documents, is included with the template in the file \texttt{ouparticle\_sample.tex}.

Here are two sample references: @Feynman1963118 {[}@Dirac1953888{]}. Bibliography will appear at the end of the document.

\hypertarget{materials-and-methods}{%
\section{Materials and methods}\label{materials-and-methods}}

An equation with a label for cross-referencing:

\begin{equation}\label{eq:eq1}
\int^{r_2}_0 F(r,\varphi){\rm d}r\,{\rm d}\varphi = [\sigma r_2/(2\mu_0)]
\int^{\infty}_0\exp(-\lambda|z_j-z_i|)\lambda^{-1}J_1 (\lambda r_2)J_0
(\lambda r_i\,\lambda {\rm d}\lambda)
\end{equation}

This equation can be referenced as follows: Eq. \ref{eq:eq1}

\hypertarget{a-subsection}{%
\subsection{A subsection}\label{a-subsection}}

A numbered list:

\begin{enumerate}
\def\labelenumi{\arabic{enumi})}
\tightlist
\item
  First point
\item
  Second point

  \begin{itemize}
  \tightlist
  \item
    Subpoint
  \end{itemize}
\end{enumerate}

A bullet list:

\begin{itemize}
\tightlist
\item
  First point
\item
  Second point
\end{itemize}

\hypertarget{results-1}{%
\section{Results}\label{results-1}}

\hypertarget{generate-a-figure.}{%
\subsection{Generate a figure.}\label{generate-a-figure.}}

\begin{Shaded}
\begin{Highlighting}[]
\FunctionTok{plot}\NormalTok{(}\DecValTok{1}\SpecialCharTok{:}\DecValTok{10}\NormalTok{,}\AttributeTok{main=}\StringTok{"Some data"}\NormalTok{,}\AttributeTok{xlab=}\StringTok{"Distance (cm)"}\NormalTok{,}\AttributeTok{ylab=}\StringTok{"Time (hours)"}\NormalTok{)}
\end{Highlighting}
\end{Shaded}

\begin{figure}[p]
\includegraphics[width=1\linewidth]{test_oup_files/figure-latex/fig1-1} \caption{This is the first figure.}\label{fig:fig1}
\end{figure}

You can reference this figure as follows: Fig. \ref{fig:fig1}.

\begin{Shaded}
\begin{Highlighting}[]
\FunctionTok{plot}\NormalTok{(}\DecValTok{1}\SpecialCharTok{:}\DecValTok{5}\NormalTok{,}\AttributeTok{pch=}\DecValTok{19}\NormalTok{,}\AttributeTok{main=}\StringTok{"Some data"}\NormalTok{,}\AttributeTok{xlab=}\StringTok{"Distance (cm)"}\NormalTok{,}\AttributeTok{ylab=}\StringTok{"Time (hours)"}\NormalTok{)}
\end{Highlighting}
\end{Shaded}

\begin{figure}[p]
\includegraphics[width=1\linewidth]{test_oup_files/figure-latex/fig2-1} \caption{This is the second figure.}\label{fig:fig2}
\end{figure}

Reference to second figure: Fig. \ref{fig:fig2}

\hypertarget{generate-a-table-using-xtable}{%
\subsection{\texorpdfstring{Generate a table using \texttt{xtable}}{Generate a table using xtable}}\label{generate-a-table-using-xtable}}

\begin{Shaded}
\begin{Highlighting}[]
\NormalTok{df }\OtherTok{=} \FunctionTok{data.frame}\NormalTok{(}\AttributeTok{ID=}\DecValTok{1}\SpecialCharTok{:}\DecValTok{3}\NormalTok{,}\AttributeTok{code=}\NormalTok{letters[}\DecValTok{1}\SpecialCharTok{:}\DecValTok{3}\NormalTok{])}

\CommentTok{\# Creates tables that follow OUP guidelines using xtable}
\FunctionTok{library}\NormalTok{(xtable) }
\FunctionTok{print}\NormalTok{(}\FunctionTok{xtable}\NormalTok{(df,}\AttributeTok{caption=}\StringTok{"This is the table caption"}\NormalTok{,}\AttributeTok{label=}\StringTok{"tab:tab1"}\NormalTok{),}
      \AttributeTok{comment=}\ConstantTok{FALSE}\NormalTok{)}
\end{Highlighting}
\end{Shaded}

\begin{table}[ht]
\centering
\begin{tabular}{rrl}
  \hline
 & ID & code \\ 
  \hline
1 &   1 & a \\ 
  2 &   2 & b \\ 
  3 &   3 & c \\ 
   \hline
\end{tabular}
\caption{This is the table caption} 
\label{tab:tab1}
\end{table}

You can reference this table as follows: Table \ref{tab:tab1}.

\hypertarget{generate-a-table-using-kable}{%
\subsection{\texorpdfstring{Generate a table using \texttt{kable}}{Generate a table using kable}}\label{generate-a-table-using-kable}}

\begin{Shaded}
\begin{Highlighting}[]
\NormalTok{df }\OtherTok{=} \FunctionTok{data.frame}\NormalTok{(}\AttributeTok{ID=}\DecValTok{1}\SpecialCharTok{:}\DecValTok{3}\NormalTok{,}\AttributeTok{code=}\NormalTok{letters[}\DecValTok{1}\SpecialCharTok{:}\DecValTok{3}\NormalTok{])}

\CommentTok{\# kable can alse be used for creating tables}
\NormalTok{knitr}\SpecialCharTok{::}\FunctionTok{kable}\NormalTok{(df,}\AttributeTok{caption=}\StringTok{"This is the table caption"}\NormalTok{,}\AttributeTok{format=}\StringTok{"latex"}\NormalTok{,}
             \AttributeTok{booktabs=}\ConstantTok{TRUE}\NormalTok{,}\AttributeTok{label=}\StringTok{"tab2"}\NormalTok{)}
\end{Highlighting}
\end{Shaded}

\begin{table}

\caption{\label{tab:tab2}This is the table caption}
\centering
\begin{tabular}[t]{rl}
\toprule
ID & code\\
\midrule
1 & a\\
2 & b\\
3 & c\\
\bottomrule
\end{tabular}
\end{table}

You can reference this table as follows: Table \ref{tab:tab2}.

\hypertarget{discussion}{%
\section{Discussion}\label{discussion}}

You can cross-reference sections and subsections as follows: Section \ref{materials-and-methods} and Section \ref{a-subsection}.

\textbf{\emph{Note:}} the last section in the document will be used as the section title for the bibliography.

\hypertarget{references}{%
\section{References}\label{references}}


\begin{notes}[Acknowledgements]
This is an acknowledgement.

It consists of two paragraphs.
\end{notes}




\end{document}
