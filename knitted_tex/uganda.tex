% Options for packages loaded elsewhere
\PassOptionsToPackage{unicode}{hyperref}
\PassOptionsToPackage{hyphens}{url}
%
\documentclass[
]{article}
\usepackage{lmodern}
\usepackage{amsmath}
\usepackage{ifxetex,ifluatex}
\ifnum 0\ifxetex 1\fi\ifluatex 1\fi=0 % if pdftex
  \usepackage[T1]{fontenc}
  \usepackage[utf8]{inputenc}
  \usepackage{textcomp} % provide euro and other symbols
  \usepackage{amssymb}
\else % if luatex or xetex
  \usepackage{unicode-math}
  \defaultfontfeatures{Scale=MatchLowercase}
  \defaultfontfeatures[\rmfamily]{Ligatures=TeX,Scale=1}
\fi
% Use upquote if available, for straight quotes in verbatim environments
\IfFileExists{upquote.sty}{\usepackage{upquote}}{}
\IfFileExists{microtype.sty}{% use microtype if available
  \usepackage[]{microtype}
  \UseMicrotypeSet[protrusion]{basicmath} % disable protrusion for tt fonts
}{}
\makeatletter
\@ifundefined{KOMAClassName}{% if non-KOMA class
  \IfFileExists{parskip.sty}{%
    \usepackage{parskip}
  }{% else
    \setlength{\parindent}{0pt}
    \setlength{\parskip}{6pt plus 2pt minus 1pt}}
}{% if KOMA class
  \KOMAoptions{parskip=half}}
\makeatother
\usepackage{xcolor}
\IfFileExists{xurl.sty}{\usepackage{xurl}}{} % add URL line breaks if available
\IfFileExists{bookmark.sty}{\usepackage{bookmark}}{\usepackage{hyperref}}
\hypersetup{
  pdftitle={Mapping the relationship between tuberculosis burden and case notifications in Uganda},
  pdfauthor={Nathaniel Henry},
  hidelinks,
  pdfcreator={LaTeX via pandoc}}
\urlstyle{same} % disable monospaced font for URLs
\usepackage{longtable,booktabs}
\usepackage{calc} % for calculating minipage widths
% Correct order of tables after \paragraph or \subparagraph
\usepackage{etoolbox}
\makeatletter
\patchcmd\longtable{\par}{\if@noskipsec\mbox{}\fi\par}{}{}
\makeatother
% Allow footnotes in longtable head/foot
\IfFileExists{footnotehyper.sty}{\usepackage{footnotehyper}}{\usepackage{footnote}}
\makesavenoteenv{longtable}
\usepackage{graphicx}
\makeatletter
\def\maxwidth{\ifdim\Gin@nat@width>\linewidth\linewidth\else\Gin@nat@width\fi}
\def\maxheight{\ifdim\Gin@nat@height>\textheight\textheight\else\Gin@nat@height\fi}
\makeatother
% Scale images if necessary, so that they will not overflow the page
% margins by default, and it is still possible to overwrite the defaults
% using explicit options in \includegraphics[width, height, ...]{}
\setkeys{Gin}{width=\maxwidth,height=\maxheight,keepaspectratio}
% Set default figure placement to htbp
\makeatletter
\def\fps@figure{htbp}
\makeatother
\setlength{\emergencystretch}{3em} % prevent overfull lines
\providecommand{\tightlist}{%
  \setlength{\itemsep}{0pt}\setlength{\parskip}{0pt}}
\setcounter{secnumdepth}{5}
\usepackage{booktabs}
\usepackage{doi}
\usepackage{float}
\usepackage{lipsum}
\usepackage{makecell}
\usepackage{url}
\usepackage{arxiv}
\ifluatex
  \usepackage{selnolig}  % disable illegal ligatures
\fi
\newlength{\cslhangindent}
\setlength{\cslhangindent}{1.5em}
\newlength{\csllabelwidth}
\setlength{\csllabelwidth}{3em}
\newenvironment{CSLReferences}[2] % #1 hanging-ident, #2 entry spacing
 {% don't indent paragraphs
  \setlength{\parindent}{0pt}
  % turn on hanging indent if param 1 is 1
  \ifodd #1 \everypar{\setlength{\hangindent}{\cslhangindent}}\ignorespaces\fi
  % set entry spacing
  \ifnum #2 > 0
  \setlength{\parskip}{#2\baselineskip}
  \fi
 }%
 {}
\usepackage{calc}
\newcommand{\CSLBlock}[1]{#1\hfill\break}
\newcommand{\CSLLeftMargin}[1]{\parbox[t]{\csllabelwidth}{#1}}
\newcommand{\CSLRightInline}[1]{\parbox[t]{\linewidth - \csllabelwidth}{#1}\break}
\newcommand{\CSLIndent}[1]{\hspace{\cslhangindent}#1}

\title{Mapping the relationship between tuberculosis burden and case notifications in Uganda}
\author{Nathaniel Henry}
\date{2021-06-09}

\begin{document}
\maketitle

\hypertarget{abstract}{%
\section{Abstract}\label{abstract}}

\lipsum[1]

\hypertarget{introduction}{%
\section{Introduction}\label{introduction}}

\lipsum[1-3]

\hypertarget{tuberculosis-tb-mapping-in-high-burden-settings}{%
\subsection{Tuberculosis (TB) mapping in high-burden settings}\label{tuberculosis-tb-mapping-in-high-burden-settings}}

\lipsum[4-5]

\hypertarget{archetypes-of-tb-data-availability}{%
\subsection{Archetypes of TB data availability}\label{archetypes-of-tb-data-availability}}

\lipsum[6-10]

\hypertarget{prior-evidence-on-the-relationship-between-tb-burden-and-case-notifications}{%
\subsection{Prior evidence on the relationship between TB burden and case notifications}\label{prior-evidence-on-the-relationship-between-tb-burden-and-case-notifications}}

\lipsum[11-14]

\hypertarget{methods}{%
\section{Methods}\label{methods}}

\lipsum[1-2]

\hypertarget{district-level-tb-prevalence-mapping}{%
\subsection{District-level TB prevalence mapping}\label{district-level-tb-prevalence-mapping}}

\lipsum[3-5]

\hypertarget{extrapulmonary-tb-correction}{%
\subsection{Extrapulmonary TB correction}\label{extrapulmonary-tb-correction}}

\lipsum[6]

\hypertarget{smoothed-time-trend-estimation-using-case-notifications}{%
\subsection{Smoothed time trend estimation using case notifications}\label{smoothed-time-trend-estimation-using-case-notifications}}

\lipsum[7-12]

\hypertarget{results}{%
\section{Results}\label{results}}

\lipsum[13]

\hypertarget{estimating-the-tb-prevalence-to-notification-ratio-pnr-by-district}{%
\subsection{Estimating the TB prevalence to notification ratio (PNR) by district}\label{estimating-the-tb-prevalence-to-notification-ratio-pnr-by-district}}

\lipsum[1-4]

\hypertarget{comparison-with-predictive-covariates-of-tb-burden-and-health-systems-access}{%
\subsection{Comparison with predictive covariates of TB burden and health systems access}\label{comparison-with-predictive-covariates-of-tb-burden-and-health-systems-access}}

\lipsum[5-10]

\hypertarget{time-trends-of-tuberculosis-prevalence-2010-2019}{%
\subsection{Time trends of tuberculosis prevalence, 2010-2019}\label{time-trends-of-tuberculosis-prevalence-2010-2019}}

\lipsum[11-13]

\hypertarget{discussion}{%
\section{Discussion}\label{discussion}}

\lipsum[14-15]

\hypertarget{model-based-adjustment-of-tb-case-notification-rates}{%
\subsection{Model-based adjustment of TB case notification rates}\label{model-based-adjustment-of-tb-case-notification-rates}}

\lipsum[1-6]

\hypertarget{nowcasting-local-variation-in-tb-prevalence-and-uncertainty}{%
\subsection{``Nowcasting'' local variation in TB prevalence and uncertainty}\label{nowcasting-local-variation-in-tb-prevalence-and-uncertainty}}

\lipsum[7-8]

\hypertarget{extension-to-other-data-contexts}{%
\subsection{Extension to other data contexts}\label{extension-to-other-data-contexts}}

\lipsum[9-11]

\hypertarget{extension-to-tb-hiv-coinfection-and-policy-implications}{%
\subsection{Extension to TB-HIV coinfection and policy implications}\label{extension-to-tb-hiv-coinfection-and-policy-implications}}

\lipsum[12-15]

\hypertarget{conclusions}{%
\subsection{Conclusions}\label{conclusions}}

\lipsum[1]

This is an example paragraph. To conclude, let's reference Figure \ref{fig:fig1} and Table \ref{tab:t1}. Here are some example citations.\textsuperscript{\protect\hyperlink{ref-Aalbers}{1},\protect\hyperlink{ref-Aguirre2009}{2}}

\hypertarget{references}{%
\section{References}\label{references}}

\hypertarget{refs}{}
\begin{CSLReferences}{0}{0}
\leavevmode\hypertarget{ref-Aalbers}{}%
\CSLLeftMargin{1. }
\CSLRightInline{Aalbers, M. B. {Do Maps Make Geography? Part 3: Reconnecting the Trace}. \emph{ACME: An International E-Journal for Critical Cartographies2} \textbf{13}, 586--588 (2014).}

\leavevmode\hypertarget{ref-Aguirre2009}{}%
\CSLLeftMargin{2. }
\CSLRightInline{Aguirre, A. {La mortalidad infantil espa{ñ}ola en el siglo XX}. \emph{Papeles de Poblacion} \textbf{61}, 75--99 (2009).}

\end{CSLReferences}

\newpage

\hypertarget{figures-and-tables}{%
\section{Figures and Tables}\label{figures-and-tables}}

\begin{figure}[!ht]

{\centering \includegraphics[width=0.7\linewidth,]{/home/nat/Documents/Dropbox/Writing/thesis/graphics/test_chapter/fig1} 

}

\caption{Example caption for a figure.}\label{fig:fig1}
\end{figure}
\newpage

\begin{table}

\caption{\label{tab:t1}Example caption for a table.}
\centering
\begin{tabular}[t]{rr}
\toprule
a & b\\
\midrule
\cellcolor{gray!6}{-1.9089035} & \cellcolor{gray!6}{1}\\
0.4934838 & 2\\
\cellcolor{gray!6}{-1.2685774} & \cellcolor{gray!6}{3}\\
0.5779566 & 4\\
\cellcolor{gray!6}{-0.5309846} & \cellcolor{gray!6}{5}\\
\addlinespace
-1.2905178 & 6\\
\cellcolor{gray!6}{0.1389421} & \cellcolor{gray!6}{7}\\
-1.4467715 & 8\\
\cellcolor{gray!6}{-0.2152460} & \cellcolor{gray!6}{9}\\
-0.4153706 & 10\\
\bottomrule
\end{tabular}
\end{table}

\end{document}
